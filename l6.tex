\chapter{Goal and self-esteem}

When a person consciously or unconsciously chooses a goal in own life, a life goal, with that he unwittingly assesses ownself. We can judge about a person's self-esteem, low or high, by what the person lives for.

If the man puts the goal of getting all the elementary material goods, that's the level of his self-esteem: as an owner of a next-generation car, an owner of luxurious cottage house, as a part of his furniture set...

If one lives to give kindness, to ease the suffering during diseases, bring joy, that's how the person judges himself: by the level of own humanity. This is a goal worth a man.

Only a vitally important goal gives you an opportunity with pride and joy. Yes, joy! Think about it: what unfortunes can face a man with a goal of increasing the kind in this world and bringing others happiness?  Helping the wrong guy? But are there many who don't need help? And if you're a doctor, well, maybe, you made a wrong diagnosis? This happens to the best doctors. In the end, you help more than you don't help. No one is safe from mistakes. But the greatest mistake, the fatal one is the wrong goal in life. No promotion makes you sad. No new stamp for your collection makes you sad. Someone has a better furniture or car than you have -- that makes you sad and how much!

Making career or buying stuff as your life goal brings many more disapointments than joy; such a man risks to lose it all. But what can you lose, when every good deed of yours makes you happy? It is important though to make sure that this kindness is your internal need, that it's managed by a smart heart, not head alone. It ought not to be just one "principle".

That's why it's necessary for the main goal in life to spread over more than one individual. It should not be focused on own successes and failures. It must be spurred by love to others, love to your family, your city, your people, your country, your universe.

Does it mean that you should be a hermit, that you should not care about yourself, that you should buy nothing and find no joy in any promotion whatsoever? Not at all! Such a man, who doesn't think about himself at all, is an unnormal phenomena and displeasing to me: this idea posesses some psychological break, some showy exaggeration of your own kindness, unselfishness, importance. This is a peculiar contempt of others and a desire to stand out.

That's why I talk only about the main life goal. There's no need to emphasize this main life goal in front of all the other people. You need to dress well (this is respectful to others), but not "better than others". You need to have good home library, but not the one bigger, that your neighbor has. And buying car is a pretty good thing for you and your family -- it's convenient. But do not turn the secondary things into the main question, and do not exhaust yourself by following your main life goal, when it's not necessary. When it is need, that's a different story. That's when we'll see what we're capable of.