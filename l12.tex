\chapter{A man ought to be cultured}

A man ought to be cultured! And what if your job doesn't need this quality? What if you couldn't get a proper education: such were the circumstances? What if the public doesn't accept you? What if being cultured makes you the black sheep of his collegues, his friends and relatives and push everyone away from you?

No, no and no! Culture is something we need at all times. We need for others, we need for ourselves.

It is so very important, and first of all it is important for those who want to live a happier and a longer life! Yes, longer! Being cultured is equal to moral health. And we need that to live longer, not just physically, but mentally as well. Bible says: “Honor your father and your mother, so that you may live long in the land.". This can be applied both to a whole nation and to one individual. This is wise.

But, first, let's define what being cultured is, and then, why is it connected to the rule of longevity.

Many think that a cultured man is someone who has read a lot, got a decent education (and even mainly in liberal arts), traveled a lot, knows some foreign languages.

In the meantime it turns out that you can meet all the requirements being a person of no culture; you can lack this things but still have a great internal culture.

Education ought not to be mixed with culture. Education is rooted in old things, culture is rooted in creating new and understanding old as new.

Moreover... Consider a truly cultured man. Take all his knowledge away from him, all his education, all his memory; make him forget everything, every classic book, every great work of art, every important event in history--but leave his sensitivity to intellectual values. Leave his love to acquiring knowledge. Leave his love to history, his aesthetic feeling, his ability to differentiate real art from cheap "stuff" made only to shock, true delight from nature, ability to understand one's character and individuality with sympathy, and thus the ability to help others. Leave him no rudeness, no apathy, no gloating, no jealousy; let him value people by their virtue. Let him show his respect to the culture of the past, his good manners, responsibility in deciding ethical questions; leave his ability to express thoughts in a rich and precise way. This is what culture looks like.

It's not in knowledge. It's in the skill of understanding others. It is all in the details: how you argue with respect, how you stay modest at the table. how you help someone without getting much attentions, how you care for nature and don't litter with roaches or bad ideas (these are such a garbage!).

I knew peasants from the Russian North, who were cultured indeed. They kept their homes clean, appreciated good songs, they could tell "byvalschina" -- stories from the past. Their life was well-ordered; they were hospitable and friendly, they respect both sorrow and joy of other people.

Culture is the ability to understand, to perceive, to be tolerant to the world and people.

We need to develop being cultured in ourselves, train it--we need to train our soul's strength, just as we need to train our physical strength. We can do that under any circumstances.

Everyone knows that our physical power is necessary to living a long life. Many fewer people understand that longevity requires training mental and spiritual powers.

The thing is a vicous and evil reaction to things around you, being rude and misunderstanding others, these are signs of one's mental and spiritual weakness, inability to live. Hustling in a crowded bus is something a weak and nervous individual does, exhausted, unable to react properly to things around him. Such a man quarrels with his neighbours, he is spiritually deaf. Such a man, an unhappy man, is blind to beauty, blind to others' good intentions, always offended, always unable to understnad others. Someone disturbing both his own life and lives of others. Moral weakness leads to physical weakness. I am not a doctor, but I have no doubts in that. An old man tells you that.

Being welcoming and kind makes you not only healthy, but beautiful. Yes, namely beautiful.

A face distorted by anger is ugly. An angry man lacks elegance in his motions; I am not talking about fake elegance, but natural one, which is much more valuable.

It is our social duty to be cultured. It is one's duty to himself. It is the pledge of personal happiness, a kinda aura that influences people around him and the man himself.

All what I talk about in this book is all one call for being cultured, for physical and moral, health, for healthy beauty. Be old as individuals and as a nation. Care for your father and mother includes care for much more: caring for all the best in your historical past. History is the father and mother of modern times, great times. What joy it is to live now!