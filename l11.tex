\chapter{On careerism}

A human being develops from the very first day of his life. Always looking forward. Always learning, making up new goals, even without fully understanding that. And how quick we know our position in this world. So quickly we learn to hold a spoon and utter our first words.

As a young man he learns too.

And one day the time comes to apply your knowledge, achieve your dreams. Maturity. We should live in the present...

But we can't stop accelarating, and so instead of learning we prefer gaining balnce. Inertia takes over. We are always headed in the future, but the future is not in real knowledge, and it's not in gaining skill, but it is only in getting yourself a better place under the sun. Contents, true substance are lost. There's no present, only empty sight towards the future. This is careerism. Personal worries make you unhappy and irresistible to watch.