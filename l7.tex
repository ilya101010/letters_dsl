\chapter{What unites people}

Layers of care. Care is what bonds people together. These bonds hold famiies together, friends together, villages together, townspeople together, citizens of countries together.

Trace anyone's life.

A human is born. Mother is the first person to care about the child. Gradually, after several days dad starts caring directly about his child (before that this care was present indeed, but pretty much abstract: parents were only getting ready, dreaming of the child).

The feeling of care towards someone else appears very early in one's life. That's especially true for girls. A little girl can't talk yet, but already babysits a doll, cares about the toy. Very little boys like to gather mushrooms and catch fish. Girls like gathering berries and mushrooms too. Children don't gather these things for just their own. They do it for the whole family. Bring home, prepare for winter.

Slowly kids become objects of an even greater care. And they themselves begin to express real and wide care: not only of their families, but of their schools (that's a choice spurred by their parents' care), of their villages, cities, countries...

Care widens and becomes more and more altruistic. Children pay for their parents' care in the early years with their own care towards old parents, when the latter can't pay anything back. This care towards the elderly, and later towards the memory of the deceased parents merges with the care about the family's and the country's historical memory.

If one has no other care than his ownself, the person is selfish.

Care unites people. It strengths the memories of the past, but dreams of the future. It's not just this feeling itself, it's the specific expressions of love, friendship, patriotism. People should care. Uncaring or carefree person is probably someone not kind, someone who loves no one.

Compassion is inherent to morality. Compassionate feel unity with the rest: the human kind, the world (not just people and peoples, but animals, plants, nature etc). The sense of compassion (or something close to that) prods us to fight for hertage sights, for their preservation. In compassion there's the understanding of your own unity with others, with your nation, people, country, universe. This is why the forgotten concept of compassions needs its full revival and development.

What a right idea: "One small step for man, one giant leap for mankind". We can find thousands of proofs to that: it's easy for one man to turn kind, but it's incredibly hard for all the people to do that. It's hard to fix the mankind, but it's easy to fix yourself. Feed a child, help an elderly man cross the road, give your seat in tram, work well, be polite and accurate etc -- these things are easy for one to start doing, but hard for everyone to do at the same time. That's why, start with yourself.

Kindness can't be silly. A kind thing to do is never a stupid one: it's unselfish, and it gives no gain or "clever results". A kind deed can be a "silly" one only if it clearly couldn't achieve the goal or it was simply "false kind", mistakenly kind, that means basically unkind. I repeat: true kind actions can not be silly, it is beyond being clever or unclever. That's what good in kindness.