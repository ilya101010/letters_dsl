\chapter{Youth is all there's in life}

When I went to school, and later university, it seemed to me, that during my "adult life" I would find myself in an enviroment completely different from the one I did then. A brand new world, where other people would surround me and nothing will stay the same... Everything turned out to be a different story. My peers stayed with me. Not all, of course: for some death came. And still my friends of youth proved themselves as the most trustworthy, long-standing ones. My social circle grew immensely, but the true friends are the old friends. Genuine friends are the ones you make in youth. I do remember, that my mother's best friends were her peers from gymnasium. And father's friends were his university classmates. And whenever I observed, one's willingness make new friends is a property steadily declining with age. Youth is the time of unity. And it's worth remembering to take care of your friendships, as true friends are there to help you in both grief and joy. In joy people need help too: help, so that you feel happiness to the core of your soul, feel and share it. Unshared happiness is no hapiness. Happiness spoils you, if you experience it alone. And when disasters come, time of losses - don't be lonely again. How unfortunate those lonely few ones are.

That's why you should stay young to a ripe old age. Value all the good, that you got then, don't wast the treasures of youth. Nothing of what you gain in youth simply disappears. Habits, acquired during early years, are the ones that stay with you during your entire life. The same goes for skills in labour. Get used to work -- and work will bring joy. How important this is for human happiness! There's no one less happy than a person that is lazy, always avoids doing work, that makes no effort.

One more thing. There's a Russian saying: "Cherish your honor from a tender age". You will remember all the deeds of your adolescence. The good ones will make you happy, the bad ones will haunt you at nights!
