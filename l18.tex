\chapter{The art of making mistakes}

I don't like watching television.

There was one particular show I could watch forever: ice dancing. But then I got tired from it, so I stopped watching the show regularly, though I might watch it occasionally. The episodes I like the most are the ones where the underdogs, those who are not within the league of the "recognized" stars, outperform expectations. Luck of amateurs or losers brings much more joy than luck of the lucky.

But what strikes me most is how a rink man corrects his movements whenever he makes a mistake. Once he falls, he instantly rises back to his dance and carries on as if nothing happened. That's what art, great art is. 

In life, however, there are many more mistakes then on ice. And just as there, you need to figure out how to rise back -- graciously. Graciousness is the key.

When you worry too much about your mistakes, when you declare that your life is over after some minor issue, it is unsettling both for you and people around you. They feel awkward and the reason is not the mistake itself, but your inability to fix it in time.

Admitting your mistakes to yourself (it is not necessary to do it publicly: it can either ) is not always easy, you need experience for this. You need experience to get back to work after a mistake as soon and as easily as possible. And other people shouldn't push you to doing this, they should persuade you to fix it. They should react like spectators during a competition: applauding the fallen who rise up with ease.