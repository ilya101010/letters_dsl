\chapter{True and false honour}

I don't like definitions and I am not always ready for them. But I can indeed point out some differences between conscience and honour.

There's one signinificant difference between conscience and honour. Conscience comes from somewhere deep in your soul, and it always a thing that cleans you. It feels as if conscience eats you alive\footnote{In Russian we say: conscience "gnaws" someone, i.e. conscience bothers heavily somoene.}. It is never wrong. It can be either silenced or too exaggerated (a very rare case). But our understanding of honour can be completely wrong, and it can do massive damage to a society. There's no more such thing as the honour of an aristocrat, but the honour of a uniform is a heavy burden still with us. Almost as if a man has dies, and the only thing left is his uniform with all medals taken away. And there's no conscientious heart beating inside.

"The honour of a uniform" is what makes directors to defend wrong and vicous projects, demand continuation of a clearly unsuccessful construction, fight with history preservation societies ("our building works are more important") etc. There are many examples of honouring a uniform in such a way.

True honour is always aligned with conscience. A false one is mirage in a desert, in a moral desert in a soul of a human (better to say, a bureaucrat).