\chapter{The big in the small}

It's hard to fit something big into something small in the physical world. But it is all different with spiritual values: smaller things can hold notions much bigger, and if you try to place something small into something big, then the big thing will simply cease to exist.

If one has a great goal, the goal has to manifest itself everywhere, even in the most insignificant things. You need to be honest when noticed and not, that's the only way to be honest in fulfilling your greater duty. The big goal captures one as a whole, influences their every single step. But it's wrong to think than you can achieve a good end with bad means.

The saying "ends justify the means" is harmful and amoral. Dostoevsky showed this well in his "Crime and Punishment". The main character of this novel, Rodion Raskolnikov, thought that by murdering the despicable old pawnbroker, he would get money he deemed neccesary for achieving his great goals and prosperity of all human kind, but instead he faces a moral breakdown. The goal is far-fetched and unachievable, the crime is real; it's horrible and nothing can justify it. You can't reach superior goals with inferior means. Be equally honest in both big and small. 

This general rule of honouring the big in the small is necessary, particularly in science. Scientific truth is the most precious thing and one should follow it in all the minutia of a research and the whole of a scientist's life. If a scientist seeks smaller goals in their field -- forcing a proof against evidence, standing for merely "more interesting results", for the pride in their effectiveness or any other form of self-advancement, that scientist is doomed to failure. Maybe this will happen not immediately, but this will happen! When they start stretching results or even fraud occurs and scientific truth takes a back seat, "science" steps away and scientist ceases to be one.

Be decisive in standing for the big. That makes everything easy and simple.
