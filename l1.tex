\chapter{The big in the small}

Physical world has its limits: you can't put something big into something small. The world of spiritual values is free of those restrainsts. Something small can encompass things much greater, while trying to contain small within something large will lead to the latter disappearing.

If one has a great goal, the goal has to manifest itself everywhere, even in the most insignificant aspects. You need to be honest when noticed and not, that's the only way to be honest in fulfilling your greater duty. The big goal captures one as a whole, influences their every single step. But it's wrong to think than you can achieve a good end with bad means.

The saying "ends justify the means" is harmful and amoral. Dostoevsky showed this well in his "Crime and Punishment". The main character of this novel, Rodion Raskolnikov, thought that by murdering the despicable old pawnbroker, he would get the money he deemed neccesary for achieving his great goals and bring prosperity to all human kind. Instead he faces a moral breakdown. The goal is far-fetched and unachievable, the crime is real; it's horrible and nothing can justify it. You can't reach superior goals with inferior means. Be equally honest in both big and small.

This general rule of honouring the big within the small is necessary, particularly in science. Scientific truth is of utmost importance. It should be sought in all the minutia of any research and throughout the whole life of a scientist. If a scientist seeks inferior goals in their field -- forcing some proof against evidence, standing for merely "more interesting results", if their effectiveness or any other form of self-advancement, that scientist is doomed to failure. Maybe this will happen not immediately, but this will happen! When they start stretching results or even fraud occurs and scientific truth takes a back seat, "science" steps away and scientist ceases to be one.

Be decisive in standing for the big. That makes everything easy and simple.
