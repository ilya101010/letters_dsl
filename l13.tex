\chapter{On good upbringing}

Not only schools and families bring children up. It is up to us to bring ourselves up too.

It is only necessary to knkow what real upbringing is.

I don't want to provide with recipes for civility, since I don't think I am perfectly civil. But there are some ideas I'd like to share with readers.

I believe that true civility is shown first and foremost at home, in your family, in how you treat your relatives.

If a man lets an unknown woman forward (even in a bus) and even opens a door for her, but at home can't help his tired wife to wash the dishes, that man is ill-bred.

If he is polite with his familiar but at home he gets angry for any reason, that man is ill-bred.

If he does not reckon with character, psychology, habits and wishes of his close ones, that man is ill-bred.

If he, being an adult, takes for granted his parents help but does not notice their need for help, that man is ill-bred.

If he turns on radio or television loudly or simply speaks loudly, while someone at home is doing their homework or reads (even his little kids), that man is ill-bred and he will make his children ill-bred.

If he pokes fun at his wife or children with no respect to their self-esteem, especially in front of strangers, that man--pardon me-- is just stupid.

A well-bred person wants to and does treat people with respect. His good beahviour is not only common and easy to him, but brings him pleasure. He is equally civil to young and old, to superior and inferior.

A well-bred never behaves "loudly", he saves others' time ("Punctuality is the politeness of kings", as the saying goes), fulfills his promises. Being well-bred means being modest and never being selfish--anywhere you are-- at home, at school, at college, at work, in a shop and in a bus.

My reader must have noticed tat I talk primarily about men, about a head of a family. This is because indeed we should let women forward... not only in doors.

But a clever woman will easily learn how exactly to always appreciate things they're entitled to by nature from men. And they know how to avoid making men to yield as much as possible. And that is much harder. This is why nature itself made sure most women (I am not talking about exceptions) have a sense of tact and natural politeness developed to a much greater extent, than men have...

There are many books on "good manners". These books explain how to behave yourself in a society, in other people's houses and at home, in theatres, at work, with the young and the old, how to choose words and clothes without making others uncomfortable. But unfortunately people gain little from these books. I think, that it happens due to the fact that such books rarely explain why you need good manners. You might think that having good manners is fake, boring, unnecessary. The road to hell is paved with good behavior.

Yes, good manners can be really external to who you are, but, generally speaking, good manners were created from the experience of many generations, they are the result of a millennial desire to be good, live in a more comfortable and beautiful way.

Where does this all come from? What is the basic rule of acquiring good manners? Is it some simple corpus of rules, "recipes" of behavior, some sermons, that are hard to remember all together?

The fundamental idea about good behavior is care, care about not spoiling other people's lives, making everyone feel well.

Do not intefere with others. Don't make noise. You can't always shut your ears to avoid noise. For example, devise dining at a table. Don't munch, don't put your fork on a plate in a resounding way, don't make noise by sucking in your soup, don't talk loudly and don't talk with your mouth full, don't wory your neighbours at the table. And don't put your elbows on a table--that might also disturb your neighbours. Make sure your clothes are tidy, that's essential to paying respect to others, guests, hosts, just strangers; do not disturb others with bad clothes. Do not exhaust everyone with endless jokes, jests and anecdotes, especially when your neighbour has heard them already. That might be embarassing. Try not to entertain others yourself, but let others tell something as well. Manners, clothes, walk, all your behavior must be reserved--and beautiful.  No beauty exhausts. It is "social". And there's always a deep meaning in so-called good manners. Your behavior only shows who you are. Do not learn manners themselves, learn what manners mean, careful attitude to the world: people, nature, animals and birds, plants, landscapes, the past of places where you live, etc.

There's no need to remember dozens of rules, remember just one: the necessity of respecting others. If you have that and resourcefulness, manners will come to you on their own, better to say, the rules of good behaviour will come, just as the desire and ability to use them.