\chapter{Be cheerful, not funny}

Content determines form, they say. It is true, but vice versa, form dertermines content. A famous American psychologist of an early 20th century William James wrote: "We cry because we're sad, but we're sad because we cry". Let's discuss the form of our behaviour, what should we get used to and what should we deem as our internal "content".

Once it was considered rather impolite to show everyone that some misfortune had happened to you, that you were sad. People had not to share their anxiety with others. You had to keep your dignity in grief, stay calm with others, avoid sinking into yourself, stay friendly and even cheerful. Staying dignified, not botherin others with your sadness, not spoilin people's mood, being friendly and joyful -- these things are a big and pure art which is helpful for living in the society and helpful to the society itself.

But what kind of cheerful is right? Noisy and intrusive joy is exhausting to people around. An always jesting young man ceases to be seen as someone with dignity in his behaviour. He becomes a buffoon. And that is the worst that can happen to anyone in society, and that means losing any humour in the end.

Don't be funny.

Don't be funny. That's not only a feature of one's behavior, that's a sign of intelligence.

Everything can be funny, even the way you dress yourself. If a man pays too much attention to matching a tie and a shirt, a shirt and a suit, that man is funny. Superfluous attention to how you look is an easy thing to spot. Man should care about dressing decently, but the care itself should be decent too. A man paying to much to attention to his outfit is unpleasant. It is another matter with women. And there should be only a hint of fashion in a man's clothes. A perfectly clean shirt, clean shoes, and a fresh but not too bright tie, that will do. Your suit may be old, it must not be untidy.

Learn the art of listening, learn the art of proper silence and making rare and appropriate jokes while talking to others. Occupy as least space as possible. That's why you shouldn't rest your elbows on the table while having supper. Stay moderate in everything, don't be intrusive even with your friendship.

Don't let your flaws worry you, if you have any. If you stutter, don't think it's that bad. Stutterers can be great speakers, who ponder every word. The best lecturer of the Moscow University, the one that's famous for its eloquent professors, historian Klyuchevsky was a stutterer. Being slightly cross-eyed makes your face look more significant. Limp does the same thing to your movements. And don't be afraid of being shy. Don't be shy of being shy: this feature of one's character is cute and not funny at all. It can become funny only if you try to get over it too much and look embarrassed of it. Be easy and patronising to your disadvantages. Don't suffer from them. Here's the worst case scenario: when an inferiority complex develops in one's soul, leading to anger, hostility and jealousy. When one loses the best thing we all have in ourselves: kindness.

There's no better music than no music at all: silence in the mountains, silence in the forest. There's no better "music in a human" than modesty and being able to stay silent, unwillingness to push yourself to the first row. There's nothing more vexing and silly in one's demeanor than snobbishness and noisiness; there's nothing more funny in a man than excessive attengtion to his suit and haircut, calculated movements, and a "fountain of jests and anecdotes", especially when those are repeated.

Be afraid of bing funny. Try to be self-effacing and quiet.

Be respectful and equal with others. Respect people around you.

These are some tips on things some may consider of secondary importance: how you behave, how you look, but also how you feel. Do not be intimidated by your physical handicaps. Stay dignified with them, that's the way to be elegant.

A young lady I know has a slight hunchback. Honestly, I can't but admire how graceful she is at those rare occasions when I meet her in museums at gallery opennings (these are the events where everyone meet -- that's why they are the holidays of culture).

One more thing. Be truthful. A liar fools himself only. He thinks people actually believe him, but they are just being polite. Lie gives itself away, it is always felt. Not only you are disgusting, you're funny.

Do not be funny. Truthfulness is beautiful, even if you confess your lying and explain your actions. That's the way to fix situation. People will respect you for that and you will show your intelligence.

Simplicity and silence, truthfulness, unpretentiousness in clothes and behaviour, that's the most attractive "form" of an individual, that becomes the most elegant "content". \footnote{I just love this passage. What a beautiful link to "Big in small"!}