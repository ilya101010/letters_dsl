\chapter{Being able to argue with dignity}

We have to argue, object, refute others, disagree a lot in our lifetimes.

Politness is best seen when one is having a discussing, argues, stands for his point of view.

An argument shows culture, logic, manners, respectfulness and--self-respect.

If they care about a win in an argument more than about truth, if they are unable to listen to their opponent, try to shout their adversaries down and intimidate with accusations, these are empty people, and their arguments are empty.

How does a clever and polite debater argue?

First of all, he listens carefully to his opponent -- someone who disagress with his opinion. Moreover, if something is unclear in his opponent's stances, he asks additional questions. And when everything is clear to him, he chooses the weakest points in his opponent's statements and clarifies if this is what his opponent means.

By doing this the debater achieves three goals: 1) the opponent is unable to object that the debater misunderstood him, that he "didn't mean that"; 2) this attentive attitude to what the opponent thinks attracts sympathy from spectators; 3) the debater gets additional time to ponder his own objections (it is important) and specify his positions in own mind.

I'd like to address the last statement specially.

If you have been polite and calm, not snobbish during the debate, you guarantee yourself a peaceful retreat.

Remember: there's nothig more beautiful in a debate than, when necessary, admit full or partial correctness of your opponent. This is how you gain respect of others. And this looks almost as an urge for compliance, you make him relent in his stances.

Of course, you can only admit your opponent being right when this does not conflict with your common understanding, your moral principles (they always must be the highest ones).

Don't flip-flop, don't give in to make your opponent like you, and, God forbid, out of cowardice or career prospects.

But conceding with dignity on something that does not make you give your (I hope, high) morals or accepting your victory without poking fun at a losing side, without celebration, without offending your opponent's ego, that is gorgeous.

One of the greatest intellectual pleasures is to follow an argument between two skilled and smart debaters.

There isn't a thing more silly in a debate than debating with no arguments. Remember, a conversation of two ladies in Gogol's "Dead Souls":\\

-- Sweetheart, it's showy!\\
-- Ah, it's not!\\
-- Ah, it is!

When there are no arguments, there are just "opinions".