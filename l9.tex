\chapter{When to take offense?}

Be offended only when someone wanted to offend you. But if there was no such intention, and the only reason for offense is an accident, why do you need to be offended?

Don't be angry then, sort out your misunderstanding, and that's it.

Well, what if someone did want to offend you? Before reacting the same way, think about this: is it worth to sink to offending someone? Offense is usually something very low, so you have to bend over to take it.

If you did make your mind to take the offense, do some calculation in your head: subtraction, division, etc. Suppose you got offended for something, you're guilty for only partly. Subtract all the things that aren't connected to you from your feeling of offense. Suppose that you got offended for some dignified purpose: divide your feeling by all the noble motives that caused an insulting remark and so on. After doing necessary calculations in your mind, you can react to the insult with much virtue; and the less importance you attach to the offense, the greater that virtue is. Some limits are present though, of course.

Anyway, superfluous resentment is a sign of a lack of intelligenec or some psychological complex. Be smart.

There's a fine English rule: take offense only when it's give to you on purpose. There's no need to be insulted by bare carelessness or forgetfulness (sometimes people are bound to have it due to age or some psychological flaws). Instead you should pay to such a forgetful person special attention -- that is beautiful and noble.

All of this relates to the cases when they offend you, but what to do when you can offend someone? Be very careful towards touchy people. Resentment is a very painful feature of character. 